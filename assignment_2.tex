\let\negmedspace\undefined
\let\negthickspace\undefined
\documentclass[journal,12pt,onecolumn]{IEEEtran}
%\documentclass[conference]{IEEEtran}
%\IEEEoverridecommandlockouts
% The preceding line is only needed to identify funding in the first footnote. If that is unneeded, please comment it out.
\usepackage{cite}
\usepackage{amsmath,amssymb,amsfonts,amsthm}
\usepackage{algorithmic}
\usepackage{graphicx}
\usepackage{textcomp}
\usepackage{xcolor}
\usepackage{txfonts}
\usepackage{listings}
\usepackage{enumitem}
\usepackage{mathtools}
\usepackage{gensymb}
\usepackage[breaklinks=true]{hyperref}
\usepackage{tkz-euclide} % loads  TikZ and tkz-base
\usepackage{listings}
%
%\usepackage{setspace}
%\usepackage{gensymb}
%\doublespacing
%\singlespacing

%\usepackage{graphicx}
%\usepackage{amssymb}
%\usepackage{relsize}
%\usepackage[cmex10]{amsmath}
%\usepackage{amsthm}
%\interdisplaylinepenalty=2500
%\savesymbol{iint}
%\usepackage{txfonts}
%\restoresymbol{TXF}{iint}
%\usepackage{wasysym}
%\usepackage{amsthm}
%\usepackage{iithtlc}
%\usepackage{mathrsfs}
%\usepackage{txfonts}
%\usepackage{stfloats}
%\usepackage{bm}
%\usepackage{cite}
%\usepackage{cases}
%\usepackage{subfig}
%\usepackage{xtab}
%\usepackage{longtable}
%\usepackage{multirow}
%\usepackage{algorithm}
%\usepackage{algpseudocode}
%\usepackage{enumitem}
%\usepackage{mathtools}
%\usepackage{tikz}
%\usepackage{circuitikz}
%\usepackage{verbatim}
%\usepackage{tfrupee}
%\usepackage{stmaryrd}
%\usetkzobj{all}
%    \usepackage{color}                                            %%
%    \usepackage{array}                                            %%
%    \usepackage{longtable}                                        %%
%    \usepackage{calc}                                             %%
%    \usepackage{multirow}                                         %%
%    \usepackage{hhline}                                           %%
%    \usepackage{ifthen}                                           %%
  %optionally (for landscape tables embedded in another document): %%
%    \usepackage{lscape}     
%\usepackage{multicol}
%\usepackage{chngcntr}
%\usepackage{enumerate}

%\usepackage{wasysym}
%\newcounter{MYtempeqncnt}
\DeclareMathOperator*{\Res}{Res}
%\renewcommand{\baselinestretch}{2}
\renewcommand\thesection{\arabic{section}}
\renewcommand\thesubsection{\thesection.\arabic{subsection}}
\renewcommand\thesubsubsection{\thesubsection.\arabic{subsubsection}}

\renewcommand\thesectiondis{\arabic{section}}
\renewcommand\thesubsectiondis{\thesectiondis.\arabic{subsection}}
\renewcommand\thesubsubsectiondis{\thesubsectiondis.\arabic{subsubsection}}

% correct bad hyphenation here
\hyphenation{op-tical net-works semi-conduc-tor}
\def\inputGnumericTable{}                                 %%

\lstset{
%language=C,
frame=single, 
breaklines=true,
columns=fullflexible
}
%\lstset{
%language=tex,
%frame=single, 
%breaklines=true
%}





\newtheorem{theorem}{Theorem}[section]
\newtheorem{problem}{Problem}
\newtheorem{proposition}{Proposition}[section]
\newtheorem{lemma}{Lemma}[section]
\newtheorem{corollary}[theorem]{Corollary}
\newtheorem{example}{Example}[section]
\newtheorem{definition}[problem]{Definition}
%\newtheorem{thm}{Theorem}[section] 
%\newtheorem{defn}[thm]{Definition}
%\newtheorem{algorithm}{Algorithm}[section]
%\newtheorem{cor}{Corollary}
\newcommand{\BEQA}{\begin{eqnarray}}
\newcommand{\EEQA}{\end{eqnarray}}
\newcommand{\define}{\stackrel{\triangle}{=}}

\bibliographystyle{IEEEtran}
%\bibliographystyle{ieeetr}


\providecommand{\mbf}{\mathbf}
\providecommand{\pr}[1]{\ensuremath{\Pr\left(#1\right)}}
\providecommand{\qfunc}[1]{\ensuremath{Q\left(#1\right)}}
\providecommand{\sbrak}[1]{\ensuremath{{}\left[#1\right]}}
\providecommand{\lsbrak}[1]{\ensuremath{{}\left[#1\right.}}
\providecommand{\rsbrak}[1]{\ensuremath{{}\left.#1\right]}}
\providecommand{\brak}[1]{\ensuremath{\left(#1\right)}}
\providecommand{\lbrak}[1]{\ensuremath{\left(#1\right.}}
\providecommand{\rbrak}[1]{\ensuremath{\left.#1\right)}}
\providecommand{\cbrak}[1]{\ensuremath{\left\{#1\right\}}}
\providecommand{\lcbrak}[1]{\ensuremath{\left\{#1\right.}}
\providecommand{\rcbrak}[1]{\ensuremath{\left.#1\right\}}}
\theoremstyle{remark}
\newtheorem{rem}{Remark}
\newcommand{\sgn}{\mathop{\mathrm{sgn}}}
\providecommand{\abs}[1]{\left\vert#1\right\vert}
\providecommand{\res}[1]{\Res\displaylimits_{#1}} 
\providecommand{\norm}[1]{\left\lVert#1\right\rVert}
%\providecommand{\norm}[1]{\lVert#1\rVert}
\providecommand{\mtx}[1]{\mathbf{#1}}
\providecommand{\mean}[1]{E\left[ #1 \right]}
\providecommand{\fourier}{\overset{\mathcal{F}}{ \rightleftharpoons}}
%\providecommand{\hilbert}{\overset{\mathcal{H}}{ \rightleftharpoons}}
\providecommand{\system}{\overset{\mathcal{H}}{ \longleftrightarrow}}
	%\newcommand{\solution}[2]{\textbf{Solution:}{#1}}
\newcommand{\solution}{\noindent \textbf{Solution: }}
\newcommand{\cosec}{\,\text{cosec}\,}
\providecommand{\dec}[2]{\ensuremath{\overset{#1}{\underset{#2}{\gtrless}}}}
\newcommand{\myvec}[1]{\ensuremath{\begin{pmatrix}#1\end{pmatrix}}}
\newcommand{\mydet}[1]{\ensuremath{\begin{vmatrix}#1\end{vmatrix}}}
%\numberwithin{equation}{section}
%\numberwithin{equation}{subsection}
%\numberwithin{problem}{section}
%\numberwithin{definition}{section}
%\makeatletter
%\@addtoreset{figure}{problem}
%\makeatother

\title{\textbf{Assignment 2}
\\ \textbf{AI1110:} Probability and Random Variables}
\author{Ch.Kushwanth
\\ AI22BTECH11006}
\begin{document}
\maketitle
\textbf{12.13.4.5 : Question.}
Find the probability \mbox{distribution} of number of successes in two tosses of die,where a success is defined as
\begin{enumerate}
\item number greater than 4
\item six appears on atleast one die
\end{enumerate}
\textbf{Ans:}
\begin{enumerate}
\item 
\begin{tabular}{|l|l|l|l|}
\hline 
\ & \ & \ & \\
\Large X & \Large 0 \  & \Large 1 \  & \Large 2 \ \\
\hline 
\ & \ & \ & \\
 \Large $\pr{X}$ &  \Large $\frac{4}{9}$ \  &  \Large $\frac{4}{9}$ \  &\Large $\frac{1}{9}$ \ \\
 \ & \ & \ & \\
\hline
\end{tabular}
\item
\begin{tabular}{|l|l|l|}
\hline 
\ & \ & \ \\
\Large Y & \Large 0 \  & \Large 1 \ \\
\hline 
\ & \ & \ \\
 \Large $\pr{Y}$ &  \Large $\frac{25}{36}$ \  &  \Large $\frac{11}{36}$ \  \\
 \ & \ & \ \\
\hline
\end{tabular}
\end{enumerate}
 \textbf{Solution:}
let, A: outcome of first throw of die,\mbox{A$\in \{1,2,3,4,5,6\}$}\\
B: outcome of first throw of die,\mbox{B$\in \{1,2,3,4,5,6\}$}\\
we know that
\begin{align}
\pr{A=i}&=\frac{1}{6}\ , i\in \{1,2,3,4,5,6\} \label{eq:v}\\
\pr{B=j}&=\frac{1}{6}\  ,j\in \{1,2,3,4,5,6\} \label{eq:n}
\end{align}
since both A,B are independent
\begin{align}
\pr{A,B}&=\pr{A}.\pr{B} \label{eq:t}
\end{align}
\begin{enumerate}
\item finding probability distribution for appearance of number greater than 4\\
let X : appearance of number greater than 4 on 2 turns,
X $\in \{ 0,1,2 \}$ \\
$\pr{X=x}$ : probability of X becoming x,x $\in \{ 0,1,2 \}$
\begin{enumerate}
\item finding $\pr{X=0}$
\begin{align}
\pr{X=0}&=\pr{A\leq4,B\leq4}\\
&=\pr{A\leq4}.\pr{B\leq4} \label{eq:q}     (\text{from } \eqref{eq:t}) \\\\
\pr{A\leq4}
&=\pr{A=1}+\pr{A=2}+\pr{A=3}+\pr{A=4} \\
&=\frac{1}{6}+\frac{1}{6}+\frac{1}{6}+\frac{1}{6} (\text{from } \eqref{eq:v})\\
\pr{A\leq4}&=\frac{4}{6}\label{eq:r}
\end{align}
similarly
\begin{align}
\pr{B\leq4}
&=\pr{B=1}+\pr{B=2}+\pr{B=3}+\pr{B=4}\\
&=\frac{1}{6}+\frac{1}{6}+\frac{1}{6}+\frac{1}{6} (\text{from } \eqref{eq:v})\\
\pr{B\leq4}&=\frac{4}{6} \label{eq:p}
\end{align}
now
\begin{align}
\pr{X=0}&=\frac{4}{6}*\frac{4}{6} (\text{from} \eqref{eq:q},\eqref{eq:p},\eqref{eq:r})\\
\pr{X=0}&=\frac{4}{9}
\end{align}
\item finding $\pr{X=1}$
\begin{align}
\pr{X=1}&=\pr{A>4}+\pr{B>4}-2.\pr{A>4,B>4}\\
&=\pr{A>4}+\pr{B>4}-2.\pr{A>4}.\pr{B>4}  (\text{from } \eqref{eq:t})\\\\
\pr{A>4}&=\pr{A=5}+\pr{A=6}\\
&=\frac{1}{6}+\frac{1}{6} (\text{from } \eqref{eq:v})\\
\pr{A>4}&=\frac{2}{6} \label{eq:e}
\end{align}
similarly,
\begin{align}
\pr{B>4}&=\pr{B=5}+\pr{B=6}\\
&=\frac{1}{6}+\frac{1}{6} (\text{from } \eqref{eq:n})\\
\pr{B>4}&=\frac{2}{6} \label{eq:y}
\end{align}
now,
\begin{align}
\pr{X=1}&=\pr{A>4}+\pr{B>4}-2.\pr{A>4,B>4}\\
&=\frac{2}{6}+\frac{2}{6}-2.\frac{2}{6}.\frac{2}{6} \ (\text{from} \eqref{eq:e}),\eqref{eq:y})\\
\pr{X=1}&=\frac{4}{9}
\end{align}
\item finding $\pr{X=2}$
\begin{align}
\pr{X=2}&=\pr{A>4,B>4}\\
&=\pr{A>4}.\pr{B>4}\  (\text{from} \eqref{eq:t})\\
&= \frac{2}{6}.\frac{2}{6} \ (\text{from} \eqref{eq:e})\\
\pr{X=2}&=\frac{1}{9}
\end{align}
\end{enumerate}
\item finding probability distribution for six to appear alteast on one die\\
let Y:appearence of six on atleast on die,Y $\in \{ 0,1\}$ \\
$\pr{Y=y}$ : probability of Y becoming y,y $\in \{ 0,1\}$
\begin{enumerate}
\item finding $\pr{Y=0}$
\begin{align}
\pr{Y=0}&=\pr{A<6,B<6}\\
&=\pr{A<6}.\pr{B<6} (\text{from } \eqref{eq:t})\\\\
\pr{A<6}&=\pr{A=1}+\pr{A=2}+\pr{A=3}+\pr{A=4}+\pr{A=5} \\
&=\frac{1}{6}+\frac{1}{6}+\frac{1}{6}+\frac{1}{6}+\frac{1}{6} (\text{from } \eqref{eq:v}) \\
\pr{A<6}&=\frac{5}{6} \label{eq:a}
\end{align}
similarly,
\begin{align}
\pr{B<6}&=\pr{B=1}+\pr{B=2}+\pr{B=3}+\pr{B=4}+\pr{B=5}\\
&=\frac{1}{6}+\frac{1}{6}+\frac{1}{6}+\frac{1}{6}+\frac{1}{6} (\text{from } \eqref{eq:n})\\
\pr{B<6}&=\frac{5}{6} \label{eq:b}
\end{align}
now,
\begin{align}
\pr{Y=0}&=\pr{A<6}.\pr{B<6}\\
&=\frac{5}{6}.\frac{5}{6}  (\text{from } \eqref{eq:a},\eqref{eq:b})\\
\pr{Y=0}&=\frac{25}{36}
\end{align}
\item finding $\pr{Y=1}$
\begin{align}
\pr{Y=1}&=\pr{A<6,B=6}+\pr{A=6,B<6}+\pr{A=6,B=6} \  (\text{from } \eqref{eq:t})\\
&=\pr{A<6}.\pr{B=6}+\pr{A=6}.\pr{B<6}+\pr{A=6}.\pr{B=6}\\
&=\frac{5}{6}.\frac{1}{6}+\frac{1}{6}.\frac{5}{6}+\frac{1}{6}.\frac{1}{6}  (\text{from }\eqref{eq:v},\eqref{eq:n},\eqref{eq:a},\eqref{eq:b} )\\
\pr{Y=1}&=\frac{11}{36}
\end{align}
\end{enumerate}
\end{enumerate}
\end{document}

