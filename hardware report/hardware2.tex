\let\negmedspace\undefined
\let\negthickspace\undefined
\documentclass[journal,12pt,onecolumn]{IEEEtran}
%\documentclass[conference]{IEEEtran}
%\IEEEoverridecommandlockouts
% The preceding line is only needed to identify funding in the first footnote. If that is unneeded, please comment it out.
\usepackage{cite}
\usepackage{amsmath,amssymb,amsfonts,amsthm}
\usepackage{algorithmic}
\usepackage{graphicx}
\usepackage{textcomp}
\usepackage{xcolor}
\usepackage{txfonts}
\usepackage{listings}
\usepackage{enumitem}
\usepackage{mathtools}
\usepackage{gensymb}
\usepackage[breaklinks=true]{hyperref}
\usepackage{tkz-euclide} % loads  TikZ and tkz-base
\usepackage{listings}
%
%\usepackage{setspace}
%\usepackage{gensymb}
%\doublespacing
%\singlespacing

%\usepackage{graphicx}
%\usepackage{amssymb}
%\usepackage{relsize}
%\usepackage[cmex10]{amsmath}
%\usepackage{amsthm}
%\interdisplaylinepenalty=2500
%\savesymbol{iint}
%\usepackage{txfonts}
%\restoresymbol{TXF}{iint}
%\usepackage{wasysym}
%\usepackage{amsthm}
%\usepackage{iithtlc}
%\usepackage{mathrsfs}
%\usepackage{txfonts}
%\usepackage{stfloats}
%\usepackage{bm}
%\usepackage{cite}
%\usepackage{cases}
%\usepackage{subfig}
%\usepackage{xtab}
%\usepackage{longtable}
%\usepackage{multirow}
%\usepackage{algorithm}
%\usepackage{algpseudocode}
%\usepackage{enumitem}
%\usepackage{mathtools}
%\usepackage{tikz}
%\usepackage{circuitikz}
%\usepackage{verbatim}
%\usepackage{tfrupee}
%\usepackage{stmaryrd}
%\usetkzobj{all}
%    \usepackage{color}                                            %%
%    \usepackage{array}                                            %%
%    \usepackage{longtable}                                        %%
%    \usepackage{calc}                                             %%
%    \usepackage{multirow}                                         %%
%    \usepackage{hhline}                                           %%
%    \usepackage{ifthen}                                           %%
  %optionally (for landscape tables embedded in another document): %%
%    \usepackage{lscape}     
%\usepackage{multicol}
%\usepackage{chngcntr}
%\usepackage{enumerate}

%\usepackage{wasysym}
%\newcounter{MYtempeqncnt}
\DeclareMathOperator*{\Res}{Res}
%\renewcommand{\baselinestretch}{2}
\renewcommand\thesection{\arabic{section}}
\renewcommand\thesubsection{\thesection.\arabic{subsection}}
\renewcommand\thesubsubsection{\thesubsection.\arabic{subsubsection}}

\renewcommand\thesectiondis{\arabic{section}}
\renewcommand\thesubsectiondis{\thesectiondis.\arabic{subsection}}
\renewcommand\thesubsubsectiondis{\thesubsectiondis.\arabic{subsubsection}}

% correct bad hyphenation here
\hyphenation{op-tical net-works semi-conduc-tor}
\def\inputGnumericTable{}                                 %%

\lstset{
%language=C,
frame=single, 
breaklines=true,
columns=fullflexible
}
%\lstset{
%language=tex,
%frame=single, 
%breaklines=true
%}





\newtheorem{theorem}{Theorem}[section]
\newtheorem{problem}{Problem}
\newtheorem{proposition}{Proposition}[section]
\newtheorem{lemma}{Lemma}[section]
\newtheorem{corollary}[theorem]{Corollary}
\newtheorem{example}{Example}[section]
\newtheorem{definition}[problem]{Definition}
%\newtheorem{thm}{Theorem}[section] 
%\newtheorem{defn}[thm]{Definition}
%\newtheorem{algorithm}{Algorithm}[section]
%\newtheorem{cor}{Corollary}
\newcommand{\BEQA}{\begin{eqnarray}}
\newcommand{\EEQA}{\end{eqnarray}}
\newcommand{\define}{\stackrel{\triangle}{=}}

\bibliographystyle{IEEEtran}
%\bibliographystyle{ieeetr}


\providecommand{\mbf}{\mathbf}
\providecommand{\pr}[1]{\ensuremath{\Pr\left(#1\right)}}
\providecommand{\qfunc}[1]{\ensuremath{Q\left(#1\right)}}
\providecommand{\sbrak}[1]{\ensuremath{{}\left[#1\right]}}
\providecommand{\lsbrak}[1]{\ensuremath{{}\left[#1\right.}}
\providecommand{\rsbrak}[1]{\ensuremath{{}\left.#1\right]}}
\providecommand{\brak}[1]{\ensuremath{\left(#1\right)}}
\providecommand{\lbrak}[1]{\ensuremath{\left(#1\right.}}
\providecommand{\rbrak}[1]{\ensuremath{\left.#1\right)}}
\providecommand{\cbrak}[1]{\ensuremath{\left\{#1\right\}}}
\providecommand{\lcbrak}[1]{\ensuremath{\left\{#1\right.}}
\providecommand{\rcbrak}[1]{\ensuremath{\left.#1\right\}}}
\theoremstyle{remark}
\newtheorem{rem}{Remark}
\newcommand{\sgn}{\mathop{\mathrm{sgn}}}
\providecommand{\abs}[1]{\left\vert#1\right\vert}
\providecommand{\res}[1]{\Res\displaylimits_{#1}} 
\providecommand{\norm}[1]{\left\lVert#1\right\rVert}
%\providecommand{\norm}[1]{\lVert#1\rVert}
\providecommand{\mtx}[1]{\mathbf{#1}}
\providecommand{\mean}[1]{E\left[ #1 \right]}
\providecommand{\fourier}{\overset{\mathcal{F}}{ \rightleftharpoons}}
%\providecommand{\hilbert}{\overset{\mathcal{H}}{ \rightleftharpoons}}
\providecommand{\system}{\overset{\mathcal{H}}{ \longleftrightarrow}}
	%\newcommand{\solution}[2]{\textbf{Solution:}{#1}}
\newcommand{\solution}{\noindent \textbf{Solution: }}
\newcommand{\cosec}{\,\text{cosec}\,}
\providecommand{\dec}[2]{\ensuremath{\overset{#1}{\underset{#2}{\gtrless}}}}
\newcommand{\myvec}[1]{\ensuremath{\begin{pmatrix}#1\end{pmatrix}}}
\newcommand{\mydet}[1]{\ensuremath{\begin{vmatrix}#1\end{vmatrix}}}
%\numberwithin{equation}{section}
%\numberwithin{equation}{subsection}
%\numberwithin{problem}{section}
%\numberwithin{definition}{section}
%\makeatletter
%\@addtoreset{figure}{problem}
%\makeatother


\title{Lab Report - Construction of a Single-Digit Random Number Generator Using Electronic Components}
\author{Chitipotu Kushwanth\\AI22btech11006}
\date{\today}

\begin{document}
\maketitle

\section{Introduction}
The objective of this lab experiment was to design and build a random number generator that generates a single-digit random number using a breadboard and various electronic components. The key components utilized in this project included a breadboard, seven-segment display, 7474 flip-flop, 7447 decoder, 7684 clock, 555 IC, resistors (1kohm, 1mohm), capacitors (100nf, 10nf), and jump wires. This report presents a concise description of the circuit, along with an output image and a block diagram, to illustrate the construction and functionality of the single-digit random number generator.

\section{List of Components}
The following components were used in the construction of the single-digit random number generator:\\
\begin{tabular}{|l|l|l|}
\hline
component&value&quantity\\
\hline
Breadboard & & 1\\
\hline
Seven-segment display& common anode& 1\\
\hline
 flip-flop& 7474 &2\\
 \hline
 decoder& 7447 &1\\
 \hline
X-OR GATE& 7684 &1 \\
\hline
555 IC& &1\\
\hline
 Resistor& 1kohm &1\\
 \hline
 Resistor& 1mohm &1\\
 \hline
 Capacitor& 100nf &1\\
 \hline
 Capacitor&10nf &1\\
 \hline
 Jump wires& &20\\
 \hline
\end{tabular}
   
\section{Circuit Description}
The single-digit random number generator circuit was designed to generate a random number ranging from 0 to 9 using a combination of flip-flops, decoders, clocks, and other electronic components. The circuit operation can be summarized as follows:

\begin{itemize}
    \item The 555 IC was configured as an astable multivibrator, generating a continuous stream of clock pulses. The resistor R1 (1kohm) and capacitor C1 (10nf) were carefully selected to determine the frequency of the clock pulses. These components were chosen to provide a suitable timing range for the random number generation process while ensuring that the output is limited to a single digit.
    
    \item The clock pulses were then fed into the 7684 clock module, which acted as a clock divider. This module divided the input clock frequency to produce slower clock cycles, allowing for precise control and synchronization within the circuit.
    
    \item The slower clock cycles were directed to the clock input of the 7474 flip-flop. This flip-flop, together with its associated capacitor C2 (100nf), provided buffering and stabilization for reliable operation. Capacitor C2 filtered any potential noise from the clock signal, ensuring stable voltage levels within the flip-flop.
    
    \item The outputs of the flip-flop were connected to the inputs of the 7447 decoder. This decoder, in combination with the resistor R2 (1kohm), decoded the binary outputs from the flip-flop into the appropriate signals for driving the seven-segment display. The resistor R2 was used to limit the current flowing through the decoder's input, preventing any damage to the components.
    
    \item Finally, the seven-segment display was connected to the decoder's outputs. It visually represented the generated random number in decimal format, displaying a single digit ranging from 0 to 9. The decoder and display were configured to ensure that only one digit would be shown at a time.
\end{itemize}

By employing this circuit configuration, the random number generator was able to reliably generate single-digit random numbers based on the timing of the clock pulses and the state of the flip-flop outputs.

\section{Output Image}
images with labels (1),(2).(3)
\begin{figure}[ht]
 \centering
 \includegraphics[width=\textwidth]{/home/kushwanth/Downloads/photo1684411820 (1).jpeg}
 \label{fig:tat}
 \caption{output 1}
 \end{figure}
 
 \begin{figure}[h]
 \centering
 \includegraphics[width=\textwidth]{/home/kushwanth/Downloads/photo1684411820 (2).jpeg}
 \label{fig:tit}
 \caption{output 2}
 \end{figure}
 
 \begin{figure}[h]
 \centering
 \includegraphics[width=\textwidth]{/home/kushwanth/Downloads/photo1684411820.jpeg}
 \label{fig:tot}
 \caption{output 3}
 \end{figure}
 
\section{Block Diagram}
picture with label (4)
\begin{figure}[h]
 \centering
 \includegraphics[width=\textwidth]{/home/kushwanth/photo1684410916.jpeg}
 \label{fig:tpt}
 \caption{block diagram}
 \end{figure}
\section{Conclusion}
This lab experiment successfully demonstrated the construction of a single-digit random number generator using a breadboard and a combination of electronic components. The designed circuit, consisting of the 555 IC, 7684 clock, 7474 flip-flop, 7447 decoder, and seven-segment display, effectively generated random numbers in the range of 0 to 9. The utilization of resistors and capacitors ensured stable operation, with only two resistors being used in the circuit. The experiment showcased the practical application of electronic components in generating random numbers and provided valuable hands-on experience in circuit construction and design.
\end{document}
