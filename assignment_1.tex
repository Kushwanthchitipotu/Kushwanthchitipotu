\let\negmedspace\undefined
\let\negthickspace\undefined
\documentclass[journal,12pt,twocolumn]{IEEEtran}
%\documentclass[conference]{IEEEtran}
%\IEEEoverridecommandlockouts
% The preceding line is only needed to identify funding in the first footnote. If that is unneeded, please comment it out.
\usepackage{cite}
\usepackage{amsmath,amssymb,amsfonts,amsthm}
\usepackage{algorithmic}
\usepackage{graphicx}
\usepackage{textcomp}
\usepackage{xcolor}
\usepackage{txfonts}
\usepackage{listings}
\usepackage{enumitem}
\usepackage{mathtools}
\usepackage{gensymb}
\usepackage[breaklinks=true]{hyperref}
\usepackage{tkz-euclide} % loads  TikZ and tkz-base
\usepackage{listings}
%
%\usepackage{setspace}
%\usepackage{gensymb}
%\doublespacing
%\singlespacing

%\usepackage{graphicx}
%\usepackage{amssymb}
%\usepackage{relsize}
%\usepackage[cmex10]{amsmath}
%\usepackage{amsthm}
%\interdisplaylinepenalty=2500
%\savesymbol{iint}
%\usepackage{txfonts}
%\restoresymbol{TXF}{iint}
%\usepackage{wasysym}
%\usepackage{amsthm}
%\usepackage{iithtlc}
%\usepackage{mathrsfs}
%\usepackage{txfonts}
%\usepackage{stfloats}
%\usepackage{bm}
%\usepackage{cite}
%\usepackage{cases}
%\usepackage{subfig}
%\usepackage{xtab}
%\usepackage{longtable}
%\usepackage{multirow}
%\usepackage{algorithm}
%\usepackage{algpseudocode}
%\usepackage{enumitem}
%\usepackage{mathtools}
%\usepackage{tikz}
%\usepackage{circuitikz}
%\usepackage{verbatim}
%\usepackage{tfrupee}
%\usepackage{stmaryrd}
%\usetkzobj{all}
%    \usepackage{color}                                            %%
%    \usepackage{array}                                            %%
%    \usepackage{longtable}                                        %%
%    \usepackage{calc}                                             %%
%    \usepackage{multirow}                                         %%
%    \usepackage{hhline}                                           %%
%    \usepackage{ifthen}                                           %%
  %optionally (for landscape tables embedded in another document): %%
%    \usepackage{lscape}     
%\usepackage{multicol}
%\usepackage{chngcntr}
%\usepackage{enumerate}

%\usepackage{wasysym}
%\newcounter{MYtempeqncnt}
\DeclareMathOperator*{\Res}{Res}
%\renewcommand{\baselinestretch}{2}
\renewcommand\thesection{\arabic{section}}
\renewcommand\thesubsection{\thesection.\arabic{subsection}}
\renewcommand\thesubsubsection{\thesubsection.\arabic{subsubsection}}

\renewcommand\thesectiondis{\arabic{section}}
\renewcommand\thesubsectiondis{\thesectiondis.\arabic{subsection}}
\renewcommand\thesubsubsectiondis{\thesubsectiondis.\arabic{subsubsection}}


\hyphenation{op-tical net-works semi-conduc-tor}
\def\inputGnumericTable{}                                 %%

\lstset{
frame=single, 
breaklines=true,
columns=fullflexible
}

\title{\textbf{Assignment 1}
\\ \textbf{AI1110:} Probability and Random Variables}
\author{Ch.Kushwanth
\\ AI22BTECH11006}


\begin{document}
\maketitle
\textbf{12.13.6.15 : Question.}
An electronic assembley consists of two subsystems,say A and B.From previous testing procedures, the following probabilities are assumed to be known:
\\ $p$(A fails)=$0.2$
\\ $p$(B alone fails)=$0.15$
\\ $p$(A and B fails)=$0.15$
\\ Evaluate the following probabilities
\\ (i) $p$(A fails given B has failed)
\\ (ii) $p$(A fails alone)
\\ \textbf{ans:}
\\ $p$(A fails given B has failed)=$ 0.5$ 
\\ $p$(A fails alone)=$0.05$
\\ \textbf{Solution:}
\\ let:
\\ \textbf{A} represent when subsystem A works
\\Similary \textbf{B} represent when subsystem B works
\\ \textit{Given in question},
\\Probability that \textbf{A} fails $p(\textbf{A}^\mathsf{c}) = 0.2$
\\Probability that \textbf{B} fails alone $p(\textbf{A} \cap \textbf{B} ^\mathsf{c}t)=0.15$
\\Probability that both \textbf{A} and \textbf{B} fail $ p(\textbf{A} ^\mathsf{c} \cap \textbf{B} ^\mathsf{c})=0.15$
\\ \textit{Now to find},
\\Probability that both \textbf{A} fails given \textbf{B} has failed $p(\textbf{A} ^\mathsf{c}|\textbf{B} ^\mathsf{c})$ and \mbox{Probability} that \textbf{A} fails alone $p(\textbf{B}\cap\textbf{A}^\mathsf{c})$
\linebreak 
\\ \textsl{To obtain }$p(\textbf{A} ^\mathsf{c}|\textbf{B} ^\mathsf{c})$
\\ \textit{we know that},
\\ $p(\textbf{A} ^\mathsf{c}|\textbf{B} ^\mathsf{c})=p( \textbf{A} ^\mathsf{c} \cap \textbf{B} ^\mathsf{c} ) / p(\textbf{B}^\mathsf{c})$
\\ \textsl{to obtain} $p(\textbf{B} ^\mathsf{c} ) $
\\let us use $p(\textbf{A} \cap \textbf{B} ^\mathsf{c}) \ and\  p(\textbf{A} ^\mathsf{c} \cap \textbf{B}^\mathsf{c})$
\\ \textit{we know that},
\\ $p(\textbf{B} ^\mathsf{c} )= p(\textbf{A} \cap \textbf{B} ^\mathsf{c}) + p(\textbf{A} ^\mathsf{c} \cap \textbf{B} ^\mathsf{c})
\\              =0.15 +0.15
\\ p(\textbf{B} ^\mathsf{c} )= 0.3$
\\ now we have $p(\textbf{B} ^\mathsf{c} )$ we can find $p(\textbf{A} ^\mathsf{c}|\textbf{B} ^\mathsf{c})$
\\ $p(\textbf{A} ^\mathsf{c}|\textbf{B} ^\mathsf{c})= 0.15/0.3$
\\ $p(\textbf{A} ^\mathsf{c}|\textbf{B} ^\mathsf{c})$ = $0.5$
\linebreak
\\ \textit{similiarly},
\\ To obtain $p(\textbf{B} \cap \textbf{A} ^\mathsf{c})$ we have to use $p(\textbf{A} ^\mathsf{c} \cap \textbf{B} ^\mathsf{c})$ and $p(\textbf{A}^\mathsf{c})$
\\ \textit{we know that},
\\ $p(\textbf{B} \cap \textbf{A} ^\mathsf{c})=p(\textbf{A}^\mathsf{c})-p(\textbf{A} ^\mathsf{c} \cap \textbf{B} ^\mathsf{c})
\\   =0.2-0.15
\\ p(\textbf{B} \cap \textbf{A} ^\mathsf{c})=0.05$
\end {document}