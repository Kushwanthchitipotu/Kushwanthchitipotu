\let\negmedspace\undefined
\let\negthickspace\undefined
\documentclass[journal,12pt,onecolumn]{IEEEtran}
%\documentclass[conference]{IEEEtran}
%\IEEEoverridecommandlockouts
% The preceding line is only needed to identify funding in the first footnote. If that is unneeded, please comment it out.
\usepackage{cite}
\usepackage{amsmath,amssymb,amsfonts,amsthm}
\usepackage{algorithmic}
\usepackage{graphicx}
\usepackage{textcomp}
\usepackage{xcolor}
\usepackage{txfonts}
\usepackage{listings}
\usepackage{enumitem}
\usepackage{mathtools}
\usepackage{gensymb}
\usepackage[breaklinks=true]{hyperref}
\usepackage{tkz-euclide} % loads  TikZ and tkz-base
\usepackage{listings}
%
%\usepackage{setspace}
%\usepackage{gensymb}
%\doublespacing
%\singlespacing

%\usepackage{graphicx}
%\usepackage{amssymb}
%\usepackage{relsize}
%\usepackage[cmex10]{amsmath}
%\usepackage{amsthm}
%\interdisplaylinepenalty=2500
%\savesymbol{iint}
%\usepackage{txfonts}
%\restoresymbol{TXF}{iint}
%\usepackage{wasysym}
%\usepackage{amsthm}
%\usepackage{iithtlc}
%\usepackage{mathrsfs}
%\usepackage{txfonts}
%\usepackage{stfloats}
%\usepackage{bm}
%\usepackage{cite}
%\usepackage{cases}
%\usepackage{subfig}
%\usepackage{xtab}
%\usepackage{longtable}
%\usepackage{multirow}
%\usepackage{algorithm}
%\usepackage{algpseudocode}
%\usepackage{enumitem}
%\usepackage{mathtools}
%\usepackage{tikz}
%\usepackage{circuitikz}
%\usepackage{verbatim}
%\usepackage{tfrupee}
%\usepackage{stmaryrd}
%\usetkzobj{all}
%    \usepackage{color}                                            %%
%    \usepackage{array}                                            %%
%    \usepackage{longtable}                                        %%
%    \usepackage{calc}                                             %%
%    \usepackage{multirow}                                         %%
%    \usepackage{hhline}                                           %%
%    \usepackage{ifthen}                                           %%
  %optionally (for landscape tables embedded in another document): %%
%    \usepackage{lscape}     
%\usepackage{multicol}
%\usepackage{chngcntr}
%\usepackage{enumerate}

%\usepackage{wasysym}
%\newcounter{MYtempeqncnt}
\DeclareMathOperator*{\Res}{Res}
%\renewcommand{\baselinestretch}{2}
\renewcommand\thesection{\arabic{section}}
\renewcommand\thesubsection{\thesection.\arabic{subsection}}
\renewcommand\thesubsubsection{\thesubsection.\arabic{subsubsection}}

\renewcommand\thesectiondis{\arabic{section}}
\renewcommand\thesubsectiondis{\thesectiondis.\arabic{subsection}}
\renewcommand\thesubsubsectiondis{\thesubsectiondis.\arabic{subsubsection}}

% correct bad hyphenation here
\hyphenation{op-tical net-works semi-conduc-tor}
\def\inputGnumericTable{}                                 %%

\lstset{
%language=C,
frame=single, 
breaklines=true,
columns=fullflexible
}
%\lstset{
%language=tex,
%frame=single, 
%breaklines=true
%}





\newtheorem{theorem}{Theorem}[section]
\newtheorem{problem}{Problem}
\newtheorem{proposition}{Proposition}[section]
\newtheorem{lemma}{Lemma}[section]
\newtheorem{corollary}[theorem]{Corollary}
\newtheorem{example}{Example}[section]
\newtheorem{definition}[problem]{Definition}
%\newtheorem{thm}{Theorem}[section] 
%\newtheorem{defn}[thm]{Definition}
%\newtheorem{algorithm}{Algorithm}[section]
%\newtheorem{cor}{Corollary}
\newcommand{\BEQA}{\begin{eqnarray}}
\newcommand{\EEQA}{\end{eqnarray}}
\newcommand{\define}{\stackrel{\triangle}{=}}

\bibliographystyle{IEEEtran}
%\bibliographystyle{ieeetr}


\providecommand{\mbf}{\mathbf}
\providecommand{\pr}[1]{\ensuremath{\Pr\left(#1\right)}}
\providecommand{\qfunc}[1]{\ensuremath{Q\left(#1\right)}}
\providecommand{\sbrak}[1]{\ensuremath{{}\left[#1\right]}}
\providecommand{\lsbrak}[1]{\ensuremath{{}\left[#1\right.}}
\providecommand{\rsbrak}[1]{\ensuremath{{}\left.#1\right]}}
\providecommand{\brak}[1]{\ensuremath{\left(#1\right)}}
\providecommand{\lbrak}[1]{\ensuremath{\left(#1\right.}}
\providecommand{\rbrak}[1]{\ensuremath{\left.#1\right)}}
\providecommand{\cbrak}[1]{\ensuremath{\left\{#1\right\}}}
\providecommand{\lcbrak}[1]{\ensuremath{\left\{#1\right.}}
\providecommand{\rcbrak}[1]{\ensuremath{\left.#1\right\}}}
\theoremstyle{remark}
\newtheorem{rem}{Remark}
\newcommand{\sgn}{\mathop{\mathrm{sgn}}}
\providecommand{\abs}[1]{\left\vert#1\right\vert}
\providecommand{\res}[1]{\Res\displaylimits_{#1}} 
\providecommand{\norm}[1]{\left\lVert#1\right\rVert}
%\providecommand{\norm}[1]{\lVert#1\rVert}
\providecommand{\mtx}[1]{\mathbf{#1}}
\providecommand{\mean}[1]{E\left[ #1 \right]}
\providecommand{\fourier}{\overset{\mathcal{F}}{ \rightleftharpoons}}
%\providecommand{\hilbert}{\overset{\mathcal{H}}{ \rightleftharpoons}}
\providecommand{\system}{\overset{\mathcal{H}}{ \longleftrightarrow}}
	%\newcommand{\solution}[2]{\textbf{Solution:}{#1}}
\newcommand{\solution}{\noindent \textbf{Solution: }}
\newcommand{\cosec}{\,\text{cosec}\,}
\providecommand{\dec}[2]{\ensuremath{\overset{#1}{\underset{#2}{\gtrless}}}}
\newcommand{\myvec}[1]{\ensuremath{\begin{pmatrix}#1\end{pmatrix}}}
\newcommand{\mydet}[1]{\ensuremath{\begin{vmatrix}#1\end{vmatrix}}}
%\numberwithin{equation}{section}
%\numberwithin{equation}{subsection}
%\numberwithin{problem}{section}
%\numberwithin{definition}{section}
%\makeatletter
%\@addtoreset{figure}{problem}
%\makeatother

\title{\textbf{ Lab Report: Exploring a Simple Music Player}}
\author{Ch.Kushwanth
\\ AI22BTECH11006}
\begin{document}
\maketitle
\Large{\textbf{Introduction:}}\\
The purpose of this lab report is to explore the inner workings of a simple music player created using Python. The music player allows users to play a collection of music tracks from a specified folder in a randomized order. It provides options to skip to the next song, replay the current order, reshuffle the playlist, or exit the program. This report delves into the implementation details and functionality of the music player.
\\
\begin{enumerate}
\item \large{\textbf{Implementation Details:}}\\
\begin{enumerate}
\item Library Dependencies:
\begin{enumerate}
\item The code utilizes the numpy, os, pydub, and pydub.playback libraries.
\item The random library is used for shuffling the playlist.
\item The os library is used for handling file and folder operations.
\item The pydub library is used for audio file manipulation and playback.\\
\end{enumerate}
\item Function Definitions:
\begin{enumerate}
\item play-playlist(playlist): This function forms the core of the music player. It handles the main logic for shuffling and playing the playlist, as well as user interactions and input handling.
\item get-audio-files-from-folder(folder-path): This function retrieves a list of audio files from a specified folder. It filters files based on supported formats and returns a tuple containing the file path and its corresponding AudioSegment object.\\
\end{enumerate}
\end{enumerate}
\item \large{\textbf{Workflow of the Music Player:}}\\
\begin{enumerate}
\item Playlist Creation:
\begin{enumerate}
\item The user specifies the folder path containing the music tracks.
\item The program scans the folder and compiles a list of compatible audio files using the get-audio-files-from-folder() function.
\item Each audio file is represented as a tuple containing the file path and its corresponding AudioSegment object.\\
\end{enumerate}
\item Shuffling and Playback:
\begin{enumerate}
\item The program shuffles the original playlist randomly using the np.random.shuffle() function.
\item It enters a continuous loop to play the songs in the shuffled order.
\item For each song, the file name is extracted from the path using the os.path.basename() function.
\item The song is played using the play() function from the pydub.playback library.\\
\end{enumerate}
\item User Interactions:
\begin{enumerate}
\item While the songs are playing, the program prompts the user for input to control the playback.
\item If the current song is the last in the playlist, the program offers options to skip to the next song, replay the current order, reshuffle the playlist, or exit the program.
\item If the current song is not the last, the user can choose to skip to the next song or continue playing.\\
\end{enumerate}
\end{enumerate}
 \item \large{\textbf{Conclusion:}}\\
 The simple music player provides an interactive way to enjoy a collection of music tracks. It allows users to shuffle the playlist, skip songs, replay the current order, reshuffle the playlist, or exit the program. By leveraging libraries such as pydub, the player handles audio file manipulation and playback seamlessly. This lab report has provided an overview of the music player's implementation details, workflow, and user interactions, offering insights into its functionality.\\
 \item \large{\textbf{outcome:}}\\
 These are some of pictures of working code.
 \begin{figure}[h]
 \centering
 \includegraphics[width=\textwidth]{/home/kushwanth/do/Screenshot from 2023-05-17 20-27-45.png}
 \caption{\large{this shows the playing of songs}}
 \end{figure}
 \begin{figure}
 \centering
 \includegraphics[width=\textwidth]{/home/kushwanth/do/Screenshot from 2023-05-17 20-28-26.png}
 \caption{\large{this shows option to skip a song or continue}}
 \end{figure}
 \begin{figure}
 \centering
 \includegraphics[width=\textwidth]{/home/kushwanth/do/Screenshot from 2023-05-17 21-44-47.png}
 \caption{\large{this shows option that appear after completing whole list}}
 \end{figure}
\end{enumerate}
\end{document}
