\let\negmedspace\undefined
\let\negthickspace\undefined
\documentclass[journal,12pt,onecolumn]{IEEEtran}
%\documentclass[conference]{IEEEtran}
%\IEEEoverridecommandlockouts
% The preceding line is only needed to identify funding in the first footnote. If that is unneeded, please comment it out.
\usepackage{cite}
\usepackage{amsmath,amssymb,amsfonts,amsthm}
\usepackage{algorithmic}
\usepackage{graphicx}
\usepackage{textcomp}
\usepackage{xcolor}
\usepackage{txfonts}
\usepackage{listings}
\usepackage{enumitem}
\usepackage{mathtools}
\usepackage{gensymb}
\usepackage[breaklinks=true]{hyperref}
\usepackage{tkz-euclide} % loads  TikZ and tkz-base
\usepackage{listings}
%
%\usepackage{setspace}
%\usepackage{gensymb}
%\doublespacing
%\singlespacing

%\usepackage{graphicx}
%\usepackage{amssymb}
%\usepackage{relsize}
%\usepackage[cmex10]{amsmath}
%\usepackage{amsthm}
%\interdisplaylinepenalty=2500
%\savesymbol{iint}
%\usepackage{txfonts}
%\restoresymbol{TXF}{iint}
%\usepackage{wasysym}
%\usepackage{amsthm}
%\usepackage{iithtlc}
%\usepackage{mathrsfs}
%\usepackage{txfonts}
%\usepackage{stfloats}
%\usepackage{bm}
%\usepackage{cite}
%\usepackage{cases}
%\usepackage{subfig}
%\usepackage{xtab}
%\usepackage{longtable}
%\usepackage{multirow}
%\usepackage{algorithm}
%\usepackage{algpseudocode}
%\usepackage{enumitem}
%\usepackage{mathtools}
%\usepackage{tikz}
%\usepackage{circuitikz}
%\usepackage{verbatim}
%\usepackage{tfrupee}
%\usepackage{stmaryrd}
%\usetkzobj{all}
%    \usepackage{color}                                            %%
%    \usepackage{array}                                            %%
%    \usepackage{longtable}                                        %%
%    \usepackage{calc}                                             %%
%    \usepackage{multirow}                                         %%
%    \usepackage{hhline}                                           %%
%    \usepackage{ifthen}                                           %%
  %optionally (for landscape tables embedded in another document): %%
%    \usepackage{lscape}     
%\usepackage{multicol}
%\usepackage{chngcntr}
%\usepackage{enumerate}

%\usepackage{wasysym}
%\newcounter{MYtempeqncnt}
\DeclareMathOperator*{\Res}{Res}
%\renewcommand{\baselinestretch}{2}
\renewcommand\thesection{\arabic{section}}
\renewcommand\thesubsection{\thesection.\arabic{subsection}}
\renewcommand\thesubsubsection{\thesubsection.\arabic{subsubsection}}

\renewcommand\thesectiondis{\arabic{section}}
\renewcommand\thesubsectiondis{\thesectiondis.\arabic{subsection}}
\renewcommand\thesubsubsectiondis{\thesubsectiondis.\arabic{subsubsection}}

% correct bad hyphenation here
\hyphenation{op-tical net-works semi-conduc-tor}
\def\inputGnumericTable{}                                 %%

\lstset{
%language=C,
frame=single, 
breaklines=true,
columns=fullflexible
}
%\lstset{
%language=tex,
%frame=single, 
%breaklines=true
%}





\newtheorem{theorem}{Theorem}[section]
\newtheorem{problem}{Problem}
\newtheorem{proposition}{Proposition}[section]
\newtheorem{lemma}{Lemma}[section]
\newtheorem{corollary}[theorem]{Corollary}
\newtheorem{example}{Example}[section]
\newtheorem{definition}[problem]{Definition}
%\newtheorem{thm}{Theorem}[section] 
%\newtheorem{defn}[thm]{Definition}
%\newtheorem{algorithm}{Algorithm}[section]
%\newtheorem{cor}{Corollary}
\newcommand{\BEQA}{\begin{eqnarray}}
\newcommand{\EEQA}{\end{eqnarray}}
\newcommand{\define}{\stackrel{\triangle}{=}}

\bibliographystyle{IEEEtran}
%\bibliographystyle{ieeetr}


\providecommand{\mbf}{\mathbf}
\providecommand{\pr}[1]{\ensuremath{\Pr\left(#1\right)}}
\providecommand{\qfunc}[1]{\ensuremath{Q\left(#1\right)}}
\providecommand{\sbrak}[1]{\ensuremath{{}\left[#1\right]}}
\providecommand{\lsbrak}[1]{\ensuremath{{}\left[#1\right.}}
\providecommand{\rsbrak}[1]{\ensuremath{{}\left.#1\right]}}
\providecommand{\brak}[1]{\ensuremath{\left(#1\right)}}
\providecommand{\lbrak}[1]{\ensuremath{\left(#1\right.}}
\providecommand{\rbrak}[1]{\ensuremath{\left.#1\right)}}
\providecommand{\cbrak}[1]{\ensuremath{\left\{#1\right\}}}
\providecommand{\lcbrak}[1]{\ensuremath{\left\{#1\right.}}
\providecommand{\rcbrak}[1]{\ensuremath{\left.#1\right\}}}
\theoremstyle{remark}
\newtheorem{rem}{Remark}
\newcommand{\sgn}{\mathop{\mathrm{sgn}}}
\providecommand{\abs}[1]{\left\vert#1\right\vert}
\providecommand{\res}[1]{\Res\displaylimits_{#1}} 
\providecommand{\norm}[1]{\left\lVert#1\right\rVert}
%\providecommand{\norm}[1]{\lVert#1\rVert}
\providecommand{\mtx}[1]{\mathbf{#1}}
\providecommand{\mean}[1]{E\left[ #1 \right]}
\providecommand{\fourier}{\overset{\mathcal{F}}{ \rightleftharpoons}}
%\providecommand{\hilbert}{\overset{\mathcal{H}}{ \rightleftharpoons}}
\providecommand{\system}{\overset{\mathcal{H}}{ \longleftrightarrow}}
	%\newcommand{\solution}[2]{\textbf{Solution:}{#1}}
\newcommand{\solution}{\noindent \textbf{Solution: }}
\newcommand{\cosec}{\,\text{cosec}\,}
\providecommand{\dec}[2]{\ensuremath{\overset{#1}{\underset{#2}{\gtrless}}}}
\newcommand{\myvec}[1]{\ensuremath{\begin{pmatrix}#1\end{pmatrix}}}
\newcommand{\mydet}[1]{\ensuremath{\begin{vmatrix}#1\end{vmatrix}}}
%\numberwithin{equation}{section}
%\numberwithin{equation}{subsection}
%\numberwithin{problem}{section}
%\numberwithin{definition}{section}
%\makeatletter
%\@addtoreset{figure}{problem}
%\makeatother

\title{\textbf{Assignment 2}
\\ \textbf{AI1110:} Probability and Random Variables}
\author{Ch.Kushwanth
\\ AI22BTECH11006}
\begin{document}
\maketitle
\textbf{12.13.4.5 : Question.}
Find the probability \mbox{distribution} of number of successes in two tosses of die,where a success is defined as
\begin{enumerate}
\item number greater than 4
\item six appears on atleast one die
\end{enumerate}
\textbf{Ans:}
\begin{enumerate}
\item 
\begin{tabular}{|l|l|l|l|}
\hline 
\ & \ & \ & \\
\Large X & \Large 0 \  & \Large 1 \  & \Large 2 \ \\
\hline 
\ & \ & \ & \\
 \Large $\pr{X}$ &  \Large $\frac{4}{9}$ \  &  \Large $\frac{4}{9}$ \  &\Large $\frac{1}{9}$ \ \\
 \ & \ & \ & \\
\hline
\end{tabular}
\vspace{3mm}
\item
\begin{tabular}{|l|l|l|}
\hline 
\ & \ & \ \\
\Large Y & \Large 0 \  & \Large 1 \ \\
\hline 
\ & \ & \ \\
 \Large $\pr{Y}$ &  \Large $\frac{25}{36}$ \  &  \Large $\frac{11}{36}$ \  \\
 \ & \ & \ \\
\hline
\end{tabular}
\end{enumerate}
 \textbf{Solution:}
 defining variables used in solution\\
 \begin{tabular}{|l|l|l|l|}
\hline
variable & what it denotes & domain of variable  \\
\hline
X & appearance of number greater than 4 on 2 turns & X$\in \{0,1,2\}$\\
\hline
Y & appearence of six on atleast on die & Y$\in \{0,1\}$\\
\hline
U & appearance of six on die & U$\in \{0,1,2\}$\\
\hline
\end{tabular}\\
\\
defining probabilities related to random variables\\
\begin{tabular}{|l|l|l|l|}
\hline
varibles & random variable probability & what it denotes & domain of microvariable\\
\hline
X & $\pr{X=x}$ & probability of X becoming x & x $\in \{ 0,1,2 \}$\\
\hline
Y &$\pr{Y=y}$ & probability of Y becoming y & y $\in \{ 0,1\}$\\
\hline
U & $\pr{U=u}$ & probability of u number of 6s appear  & u$\in \{0,1,2\}$\\
\hline
\end{tabular}\\

\begin{enumerate}
\item finding probability distribution for appearance of number greater than 4\\
$p$ denotes probability that number greater that 4 appears,$X:bin(n,p)$ 
\begin{align}
p=\frac{2}{6}(\text{as there are 2 numbers greater than  4 as outcome of die})\label{eq:k} 
\end{align}
Using binomial distribution,X:$bin(2,\frac{1}{3})$
\begin{align}
\pr{X=i}&={n \choose i}\times(p)^i\times(1-p)^{n-i}\\
\text{here } n=2,p=\frac{1}{3}  (\text{from } \eqref{eq:k})\\
\pr{X=i}&={2 \choose i}\times \brak{\frac{1}{3}}^i\times\brak{1-\frac{1}{3}}^{2-i} \label{eq:r}
\end{align}
\begin{enumerate}
\item finding $\pr{X=0}$
\begin{align}
\pr{X=0}&=\frac{2!}{(0!)\times((2-0)!)}\times\brak{\frac{1}{3}}^0\times\brak{1-\frac{1}{3}}^{2-0} (\text{from } \eqref{eq:r})\\
&= \frac{2}{(1)\times(2)}\times(1)\times\brak{\frac{2}{3}}^{2}    \\
\pr{X=0}&=\frac{4}{9}
\end{align}
\item finding $\pr{X=1}$
\begin{align}
\pr{X=1}&=\frac{2!}{1!\times(2-1)!}\times\brak{\frac{1}{3}}^{1}\times\brak{\frac{2}{3}}^{2-1}(\text{from } \eqref{eq:r})\\
&=\frac{2}{(1)\times(1)}\times\brak{\frac{1}{3}}\times\brak{\frac{2}{3}}\\
\pr{X=1}&=\frac{4}{9}
\end{align}
\item finding $\pr{X=2}$
\begin{align}
\pr{X=2}&=\frac{2!}{2!\times(2-2)!}\times\brak{\frac{1}{3}}^{2}\times\brak{\frac{2}{3}}^{2-2}(\text{from } \eqref{eq:r})\\
&= \frac{2}{(2)\times(1)}\times\brak{\frac{1}{3}}^{2}\times(1)    \\
\pr{X=2}&=\frac{1}{9}
\end{align}
\end{enumerate}
\item finding probability distribution for six to appear alteast on one die\\
$p$ denotes the probability of 6 on one throw of die.$U:bin(n,p)$
\begin{align}
 p&=\frac{1}{6} 
\end{align}
using binomial distribution,$U:bin(2,\frac{1}{6})$
\begin{align}
\pr{U=i}&={n \choose i}\times(p)^i\times(1-p)^{n-i}\\
\text{here } n=2,p=\frac{1}{6} \\
\pr{U=i}&={2 \choose i}\times\brak{\frac{1}{6}}^i\times\brak{1-\frac{1}{6}}^{2-i}\\
\pr{U=i}&=\frac{2!}{i!\times(2-i)!}\times\brak{\frac{1}{6}}^i\times\brak{\frac{5}{6}}^{2-i} \label{eq:a}
\end{align} 
\begin{enumerate}
\item finding $\pr{Y=0}$
\begin{align}
\pr{Y=0}&=\pr{U=0}\\
&=\frac{2!}{0!\times(2-0)!}\times\brak{\frac{1}{6}}^0\times\brak{\frac{5}{6}}^{2-0} (\text{from } \eqref{eq:a})\\
&=\frac{2}{1\times2}(1)\times\brak{\frac{5}{6}}^{2}\\
\pr{Y=0}&=\frac{25}{36}
\end{align}
\item finding $\pr{Y=1}$
\begin{align}
\pr{Y=1}&=p_U(1)\\
&=\pr{U=1}+\pr{U=2}\\
&=\frac{2!}{1!\times(2-1)!}\times\brak{\frac{1}{6}}^1\times\brak{\frac{5}{6}}^{2-1}+\frac{2!}{2!\times(2-2)!}\times\brak{\frac{1}{6}}^2\times\brak{\frac{5}{6}}^{2-2} (\text{from }\eqref{eq:a})\\
&=\frac{2}{(1)\times(1)}\times\brak{\frac{1}{6}}\times\brak{\frac{5}{6}}+\frac{2}{(2)\times(1)}\times\brak{\frac{1}{36}}\times(1)  \\
&=\frac{5}{36}+\frac{5}{36}+\frac{1}{36}\\
\pr{Y=1}&=\frac{11}{36}
\end{align}
\end{enumerate}
\end{enumerate}
\end{document}

